\section{Conclusion} \label{sec:Conclusion}
To conclude, we will discuss the scope of this work in brief. We have developed a number of mathematical formalisms to help solve the problem described in the introduction, and have used these to describe the motion of a door experimentally, as well as to develop computational systems to model events under ideal conditions. We have only to answer the question of which door to take, which after reading the first two pages is nearly trivial.  Whichever of equations (9) or (10) is lesser for the situation's specific case of manual door closer will correspond to their choosing the open or closed door respective, to have the easiest time passing through (or the converse to get a better workout).\par
The only caveat is that not all door dampeners work with a single modality, such as the screen door we have used in our experiments. Though the formalism is evidently descriptive of each type of dampening, it is obvious that some door dampeners, such as the one considered here, also apply their effects as some function of the angle (in this case, no effect for ~125°-45° and ~7°-0°, and some critical or overdamped effect for ~45°-7°). These will obviously vary by make and model, and a discussion on which is best engineering practice may certainly follow from this paper.