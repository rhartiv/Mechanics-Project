\section{Introduction} \label{sec:Introduction}
    Say you are walking behind someone at some distance, and the two of you come to a set of double doors. The person in front of you opens one of the doors, but is rather busy and opts not to hold the door for you. As you approach the doors, the open door is now swinging back towards you. Should you take the open, swinging door, or should you take the closed door?

    Obviously, if you can just slip through (if the person in front of you has left the door rather widely open), then you should take the open door. However if the open door is nearly closed, the angular momentum of the door will (likely) be at its greatest value. In this case, not only will you need to push the open door (nearly) as far as you would the closed door, but you would also need to account for the additional angular impulse required to change the direction of the angular momentum, and thus you should take the closed door. 

    In this paper we will discuss the work required to open the closed door versus different cases of the closing door, and develop a model based on such. Further, we will do experiment as well as a number of numerical/computational analyses.